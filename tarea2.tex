% ====== TAREA 2 MATEMATICAS APLICADAS ======
\documentclass{article}
\usepackage[utf8]{inputenc}
\usepackage[spanish]{babel}
\usepackage{amsmath, amsfonts, amssymb}
\usepackage{graphics}
\usepackage[usenames]{color}
\usepackage[text={20cm,25cm},centering,top=1.5cm,bottom=1.5cm,letterpaper,showframe=false]{geometry}
\renewcommand{\baselinestretch}{1.5}
\parindent  = 0mm
\parskip    = 4mm
\definecolor{azul}{RGB}{10,80,190}
\definecolor{negro}{RGB}{0,0,0}
\definecolor{rojo}{RGB}{190,80,10}
\definecolor{verde}{RGB}{0,120,50}

\begin{document}
    \title{Tarea 2}
    \author{Careaga Carrillo Juan Manuel\\
            Quiróz Castañeda Edgar\\
            Soto Corderi Sandra del Mar}
    \date{Miércoles 10 de octubre de 2018}
    \maketitle
    \begin{enumerate}
        % Ejercicio 1
        \item {
            Encontrar la imagen de un triángulo con vérticies $(0,0)$, $(1,1)$
            y $(0,1)$ bajo la transformación $x=u^2$ y $y=v$.

            \color{azul}
            % Respuesta
        }

        % Ejercicio 2
        \item {
            Calcular
            \[
                \iint_{D}{e^{\frac{x-2y}{3x-y}}\,dA}
            \]
            con $D$ el paralelogramo acotado por las rectas $x-2y=0$, $x-2y=$,
            $3x-y=1$ y $3x-y=8$-

            \color{azul}
            % Respuesta
        }

        % Ejercicio 3
        \item {
            Hallar el volumen del elipsoide
            \[
                \frac{x^2}{a^2}+\frac{y^2}{b^2}+\frac{z^2}{c^2}\leq 1
            \]

            \color{azul}
            % Respuesta
        }

        % Ejercicio 4
        \item {
            Hallar el área acotada por la lemniscata
            \(
                \left(x^2+y^2\right)^2=2a^2\left(x^2-y^2\right)
            \).

            \color{azul}
            % Respuesta
        }

        % Ejercicio 5
        \item {
            Evaluar la integral iterada
            \[
                \int_{0}^{2}{
                    \int_{-\sqrt{2x-x^2}}^{\sqrt{2x-x^2}}{
                        \int_{0}^{x^2+y^2}{
                            \sqrt{x^2+y^2}
                        \,dz}
                    \,dy}
                \,dx}
            \]
            y bosquejar la región de integración $W$.

            \color{azul}
            % Respuesta
        }

        % Ejercicio 6
        \item {
            Calcular la masa del sólido que se encuentra fuera de la esfera
            $x^2+y^2+^z2=1$ y dentro de la esfera $x^2+y^2+z^2=2$ suponiendo
            que la densidad en un punto $P$ es directamente proporcional al
            cuadrado de la distancia de $P$ al centro de la esfera.

            \color{azul}
            % Respuesta
        }

        % Ejercicio 7
        \item {
            Determinar los números reales $\lambda$ para los que
            \[
                \iint_D {\frac{dA}{\left(x^2+y^2\right)^\lambda}}
            \]
            es convergente, con $D$ el disco unitario con centro en el origen.

            \color{azul}
            % Respuesta
        }

        % Ejercicio 8
        \item {
            Calcular
            \[
                \iint_D {xye^{-\left(x^2+y^2\right)}\,dA}
            \]
            con $x\geq 0$ y $0\leq y\leq 1$.

            \color{azul}
            % Respuesta
        }
    \end{enumerate}
\end{document}
