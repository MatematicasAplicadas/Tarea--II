% ====== TAREA 2 MATEMATICAS APLICADAS ======
\documentclass{article}
\usepackage[utf8]{inputenc}
\usepackage[spanish]{babel}
\usepackage{amsmath, amsfonts, amssymb}
\usepackage{graphicx}
\usepackage[usenames]{color}
\usepackage[text={20cm,25cm},centering,top=1.5cm,bottom=1.5cm,letterpaper,showframe=false]{geometry}
\renewcommand{\baselinestretch}{1.5}
\parindent  = 0mm
\parskip    = 4mm
\definecolor{azul}{RGB}{10,80,190}
\definecolor{negro}{RGB}{0,0,0}
\definecolor{rojo}{RGB}{190,80,10}
\definecolor{verde}{RGB}{0,120,50}

\begin{document}
    \title{Tarea 2}
    \author{Careaga Carrillo Juan Manuel\\
            Quiróz Castañeda Edgar\\
            Soto Corderi Sandra del Mar}
    \date{Viernes 12 de octubre de 2018}
    \maketitle
    \begin{enumerate}

        % Ejercicio 1
        \item {
            Encontrar la imagen de un triángulo con vértices $(0,0)$, $(1,1)$
            y $(0,1)$ bajo la transformación $x=u^2$ y $y=v$.

            \color{azul}
          Aplicamos la transformación a varios puntos del triángulo y obtenemos
          lo siguiente:
            \begin{center}
                \includegraphics[width=6cm]{img/ejercicio1-1.png}
                \hspace{.5cm}
                \includegraphics[width=9cm]{img/ejercicio1-2.png}
        	\end{center}
            En la imagen izquierda vemos el triángulo original, y en la derecha
            vemos la imagen del triángulo bajo la transformación.
            
            Podemos observar que la transformación convirtió el lado del
            triángulo que iba de $(0,0)$ a $(1,1)$ en una curva, ya que $x=u^2$,
            el resto de los lados parecen verse no afectados, pero los puntos
            tomados si cambiaron en posición, recorriéndose, ya que al tomar
            número decimales al cuadrado obtenemos coordenadas menores a las
            originales.
	    }

        % Ejercicio 2
        \pagebreak
        \item {
            Calcular
            \[
                \iint_{D}{e^{\frac{x-2y}{3x-y}}\,dA}
            \]
            con $D$ el paralelogramo acotado por las rectas $x-2y=0$, $x-2y=$,
            $3x-y=1$ y $3x-y=8$

            \color{azul}
            El paralelogramo $D$ lo podemos ver como la siguiente imagen:
            \begin{center}
                \includegraphics[width=6cm]{img/ejercicio2.png}
         	\end{center}
             Para facilitar la integral, realizaremos un cambio de variable
             y una transformación lineal con $D$.

             Obtenemos el Jacobiano de la siguiente forma:
             \[
                 \begin{bmatrix}
                     \frac{\partial x}{\partial u}&
                     \frac{\partial x}{\partial v}\\
                     \frac{\partial y}{\partial u}&
                     \frac{\partial y}{\partial v}\\
                 \end{bmatrix}
                 =
                 \begin{bmatrix}
                     -\frac{1}{5}
                     &\frac{2}{5}\\
                     -\frac{3}{5}
                     &\frac{1}{5}\\
                 \end{bmatrix}
                 =\frac{1}{5}
             \]
             Para la transformación, tomamos $u = x-2y$ y $v= 3x-y$.
            
             Despejando con un sistema de ecuaciones de $u$ y $v$, tenemos que
             $x = -(\frac{u-2v}{5})$ y  $y = -(\frac{3u-v}{5})$

             De ahí, si aplicamos la transformación en los lados del
             paralelogramo tenemos:
             $$T(x-2y=0) \Rightarrow (\frac{-u+2v}{5}) + 2(\frac{3u-v}{5}) =
             \frac{-u+2v+6u-2v}{5} = \frac{5u}{5} \Rightarrow u = 0$$
             $$T(x-2y=4) \Rightarrow (\frac{-u+2v}{5}) + 2(\frac{3u-v}{5}) =
             \frac{-u+2v+6u-2v}{5} = \frac{5u}{5} \Rightarrow u = 4$$
             $$T(3x-y=1) \Rightarrow -3(\frac{u-2v}{5}) + (\frac{3u-v}{5}) =
             \frac{-3u+6v+3u-v}{5} = \frac{5v}{5} \Rightarrow v = 1$$
             $$T(3x-y=8) \Rightarrow -3(\frac{u-2v}{5}) + (\frac{3u-v}{5}) =
             \frac{-3u+6v+3u-v}{5} = \frac{5v}{5} \Rightarrow v = 8$$

             La imagen del paralelogramo bajo la transformación es como la
             siguiente imagen, vemos que es un cuadrado, así que los límites
             de $u$ y $v$ son sencillos.
             \begin{center}
                 \includegraphics[width=5cm]{img/ejercicio2-1.png}
         	\end{center}
             De ahí, podemos resolver a nuestra integral doble como:
             \[
                \int_{1}^{8}{
                    \int_{0}^{4}{
                        \frac{1}{5} e^{\frac{u}{v}}
                    \,du}
                \,dv}
            \]
             Usando integración por sustitución tomando $s =\frac{u}{v}$,
             resolvemos directamente $du$
             \[
                = \frac{1}{5} \int_{1}^{8}{
                    v\left(e^{\frac{4}{v}} -1\right)
                \,dv}
            \]
            \[
                \int_{1}^{8}{
                    v\left(e^{\frac{4}{v}} -1\right)
                \,dv}
                =
                \int_{1}^{8}{
                    ve^{\frac{4}{v}}
                \,dv}
                -
                \int_{1}^{8}{v\,dv}
            \]
            \[
                \int_{1}^{8}{v\,dv}
                = \frac{v^2}{2}\Big|_1^8
                = \frac{63}{2}
            \]
             Usando integración por sustitución tomando $t =\frac{4}{v}$,
             $dt = -\frac{v^2}{4}dv$ y $v= \frac{4}{t}$
             \[
                \int_{1}^{8}{ve^{\frac{4}{v}}\,dv}
                = -16\int_{1}^{8}{\frac{e^{t}}{t^3}\,dt}
            \]
            Usando integración por partes donde $u = e^t$ y
            $dv = \frac{1}{t^3}$
            \[
                \int_{1}^{8}{\frac{e^{t}}{t^3}\,dt}
                = -\frac{e^t}{2t^2} + \int{-\frac{e^t}{2t^2}\,dt}
            \]
            Nuevamente por integración por partes donde $u = e^t$ y
            $dv = \frac{1}{2t^2}$
            \[
                \int{-\frac{e^t}{2t^2}\,dt}
                = -\frac{1}{2} \left(
                    -\frac{e^t}{t} + \int{\frac{e^t}{t}\,dt}
                \right)
            \]
            Finalmente, con la definición $Ei(t) = \int\frac{e^t}{t}dt$,
            resolvemos el resto de integral. También regresamos a los valores
            originales
            \[
                \int_{1}^{8}{ve^{\frac{4}{v}}\,dv}
                = \left[
                    -16\left(
                        -\frac{1}{32}e^{\frac{4}{v}}v^2
                        +\frac{1}{2}\left(-\frac{1}{4}\right)e^{\frac{4}{v}}v
                        +Ei\left(\frac{4}{v}\right)
                    \right)
                \right]_1^8
            \]
            De todo lo anterior, el resultado sería:
            \[
                \int_{1}^{8}{
                    \int_{0}^{4}{
                        -\frac{7}{25}e^{\frac{u}{v}}
                    \,du}
                \,dv}
                =\left(\frac{1}{5}\right)\left(48\sqrt{e}
                -8Ei\left(\frac{1}{2}\right)
                -32
                -\frac{5e^4 - 16Ei(4)}{2}\right)
            \]
        }

        % Ejercicio 3
        \item {
            Hallar el volumen del elipsoide
            \[
                \frac{x^2}{a^2}+\frac{y^2}{b^2}+\frac{z^2}{c^2}\leq 1
            \]

            \color{azul}
            Hallar el volumen de forma típica nos daría una integral triple
            bastante compleja, así que para auxiliarnos, transformamos la
            elipsoide a coordenas esféricas y realizamos un cambio de variables.
            \begin{center}
                \includegraphics[width=10cm]{img/elipsoide.png}
        	\end{center}
            Tomamos la siguiente transformación, basada en coordenadas
            esféricas:
            \begin{align*}
                x &= ar\sen\gamma\cos{\theta}\\
                y &= br\sen\gamma\sen{\theta}\\
                z &= cr\cos\gamma
            \end{align*}

            El jacobiano sería la multiplicación de $abc$ con el jacobiano de
            coordenadas esféricas que vimos en clase es $r^2\sen{\gamma}$. De
            ahí, el jacobiano es $abcr^2\sen{\gamma}$

            El elipsoide es simétrico respecto al origen, los ejes y los planos
            de coordenadas. Además las secciones con planos paralelos a los
            coordenados son elipses (caso particular: circunferencia). De esta
            forma si vemos al elipsoide en coordenas esféricas podemos
            representar su volumen de esta manera:
        	\[
                \int_{0}^{1}{
                    \int_{0}^{2\pi}{
                        \int_{0}^{\pi}{
                            abcr^2\sen\gamma
                        \,d\gamma}
                    \,d\theta}
                \,dr}
            \]
            Debido a que las variables son independientes una de otra podemos
            resolver la integral así:
            \[
                (abc)
                \left(\int_{0}^{1}r^2 dr\right)
                \left(\int_{0}^{2\pi} d\theta\right)
                \left(\int_{0}^{\pi}\sen{\gamma} d\gamma\right)
                =
                (abc)
                \left(\frac{r^3}{3}\Big|_0^1\right)
                \left(\theta\Big|_0^{2\pi}\right)
                \left(-\cos{\gamma}\Big|_0^{\pi}\right)
                =
                (abc)\left(\frac{1}{3}\right)
                \left(2\pi\right)(2)
                =
                \frac{4}{3}\pi abc
            \]
            Por lo tanto, el volumen del elipsoide es
            $\displaystyle \frac{4}{3}\pi abc$.
        }

        % Ejercicio 4
        \item {
            Hallar el área acotada por la lemniscata
            \(
                \left(x^2+y^2\right)^2=2a^2\left(x^2-y^2\right)
            \).

            \color{azul}
          	Para pasar de coordenadas rectangulares a polares, se tiene que
            $r^2 = x^2+y^2$, $x = r\cos \theta$ y  $y = r\sen \theta$.
            Entonces, la ecuación de la lemniscata se puede reescribir como
            \begin{align*}
                (r^2)^2 &= 2a^2(r^2 \cos^2 \theta - r^2 \sen^2 \theta)\\
                (r^2)^2 &= 2a^2r^2(\cos^2 \theta - \sen^2 \theta)\\
                (r^2)^2 &= 2a^2r^2\cos2\theta\\
                r^2 &= 2a^2\cos2\theta\\
                r &= a\sqrt{2\cos2\theta}
            \end{align*}
            Para calcular el área baja un curva polar en un rango
            $\theta \in[\alpha, \beta]$, se tiene que definir una
            suma de Riemann. En lugar de utilizar rectángulos de altura $f(x*)$
            ,base $\Delta x$ y área $f(x*)\Delta x$ en un rango $x\in[a, b]$,
            se tienen que usar fragmentos de círculo de radio $r(\theta *)$
            y de ángulo $\Delta \theta$.\\
            El área de un fragmento de círculo de radio $r$ y ángulo $\theta$
            es el área total del círculo $\pi r^2$ por la fracción del círculo
            que representa el ángulo $\frac{\theta}{2\pi}$. Por lo que es área
            sería $\frac{\pi r ^2\theta}{2\pi} = \frac{r^2\theta}{2}$.\\
            Entonce la suma de Riemann sería
            \[\sum_{i = 1}^n {\frac{r(\theta *)^2}{2}}\Delta \theta\]
            Y el área bajo la curva polar sería
            \[
                \lim_{n \to \infty}\sum_{i = 1}^n {\frac{r(\theta *)}{2}}\Delta \theta
                = \frac{1}{2}\int_\beta^\alpha {r(\theta)^2}d\theta
            \]
            Volviendo a la lemniscata, notemos que es simétrica respecto al eje
            $x$ y al eje $y$, por lo que se puede considerar sólo el primer
            cuandrante y multiplicar lo obtenido por 4 para tener el área
            total.

            Además, la función que se tiene del radio $r = a\sqrt{2\cos2\theta}$
            no está definida para valores negativos de $\cos2\theta$.
            \[
                \cos2\theta \geq 0 \implies 0 \leq 2\theta \leq \frac{\pi}{2}
                \implies 0 \leq \theta \leq \frac{\pi}{4}
            \]
            Entonces los límites de integración son 0 y $\frac{\pi}{4}$.\\
            El área de un cuarto de la lemniscata es

            \begin{align*}
                \frac{1}{2}\int_0^{\frac{\pi}{4}}{r(\theta)^2 d\theta}
                &= \frac{1}{2}\int_0^{\frac{\pi}{4}}{(a\sqrt{2\cos2\theta})^2 d\theta}
                = \frac{1}{2}\int_0^{\frac{\pi}{4}}{a^2 2\cos2\theta d\theta}
                = \frac{1}{2}\cdot 2a^2 \int_0^{\frac{\pi}{4}}{\cos2\theta d\theta}\\
                &= a^2 \Big (\frac{\sen 2\theta}{2} \Big |_0^{\frac{\pi}{4}} \Big )
                = \frac{a^2}{2} (\sen (2\cdot\frac{\pi}{4}) - \sen (2\cdot0))
                = \frac{a^2}{2} (1-0) = \frac{a^2}{2}
            \end{align*}
            Por lo que el área total de la lemniscata es
            $4 \cdot \frac{a^2}{2} = 2a^2$
        }

        % Ejercicio 5
        \item {
            Evaluar la integral iterada
            \[
                \int_{0}^{2}{
                    \int_{-\sqrt{2x-x^2}}^{\sqrt{2x-x^2}}{
                        \int_{0}^{x^2+y^2}{
                            \sqrt{x^2+y^2}
                        \,dz}
                    \,dy}
                \,dx}
            \]
            y bosquejar la región de integración $W$.

            \color{azul}
            Pasando a coordenadas cilíndricas.\\
            Se tiene que $r^2 = x^2 + y^2$, $x = r\cos(\theta)$,
            $y = r\sen(\theta)$
            y $z = z$.\\
            La función $\sqrt{x^2+y^2}$ se vuelve $r \cdot r = r^2$, por el
            jacobiano.\\
            Los límites de $z$ se vuelven 0 y $r^2$.

            Para los límites de $r$ y de $\theta$ notemos que
            \begin{align*}
                &-\sqrt{2x-x^2} \leq y \leq \sqrt{2x-x^2}
                \implies y^2 \leq 2x-x^2 \implies y^2 + x^2 - 2x \leq 0\\
                &\implies y^2 + x^2 - 2x + 1 \leq 1
                \implies (y - 0)^2 + (x - 1)^2 \leq 1^2
            \end{align*}
            Por lo que los límites de $y$ son un círculo de radio $1$ centrado
            en $(1, 0)$.\\
            Y como los límites en $x$ son precisamente 0 y 2, entonces el área
            de integración definida por lo límites de $x$ y $y$ son todos los
            puntos dentro de ese círculo.\\
            Hay que reescribir ese círculo como función polar.
            \begin{align*}
                (y-0)^2+(x-1)^2=1^2
                &\implies y^2+x^2-2x+1 = 1\\
                &\implies y^2+x^2 = 2x\\
                &\implies r^2 = 2r\cos(\theta)\\
                &\implies r = 2\cos(\theta)
            \end{align*}
            Entonces los límites del radio son 0 y $2\cos(\theta)$.\\
            Faltan los rangos de $\theta$. Como el radio es una distancia,
            siempre es positivo. Entonces
            \[
                \cos(\theta) \geq 0 \implies 0 \leq \theta \leq \frac{\pi}{2},
                \frac{3\pi}{2} \leq \theta \leq 2 \pi
            \]
            O más simplemente $-\frac{\pi}{2} \leq \theta \leq \frac{\pi}{2}$.\\
            Entonces, la integral cilíndrica es
            \[
                \int_{-\frac{\pi}{2}}^{\frac{\pi}{2}}{
                    \int_{0}^{2\cos(\theta)}{
                        \int_{0}^{r^2}{
                            r^2
                        \,dz}
                    \,dr}
                \,d\theta}
            \]

            Primero
            \[
                \int_{0}^{r^2}{r^2dz} = (r^2 \cdot z) \Big |_0^{r^2}
                = r^2 (r^2 - 0) = r^4
            \]
            Luego
            \[
                \int_{0}^{2\cos(\theta)}{r^4dr}
                = \frac{r^5}{5} \Big |_{0}^{2\cos(\theta)}
                = \frac{1}{5} ((2\cos(\theta))^5 - 0^5)
                = \frac{32\cos^5(\theta)}{5}
            \]
            Finalmente
            \begin{align*}
                &\int_{-\frac{\pi}{2}}^{\frac{\pi}{2}}{\frac{32\cos^5(\theta)}{5}d\theta}
                = \frac{32}{5} \int{\cos^5(\theta)d\theta}
                = \frac{32}{5} \int{(\cos^2(\theta))^2\cos(\theta)d\theta}\\[0.3cm]
                &= \frac{32}{5} \int{(1-\sen^2(\theta))^2\cos(\theta)d\theta}
                = \frac{32}{5} \int{(1-u^2)^2du}
                = \frac{32}{5} \int{(1-2u^2+u^4)du}\\[0.3cm]
                &= \frac{32}{5} (u - \frac{2u^3}{3} + \frac{u^5}{5})
                = \frac{32}{5} (\sen(\theta) - \frac{2\sen(\theta)^3}{3}
                + \frac{\sen(\theta)^5}{5}) \Big |_{-\frac{\pi}{2}}^{\frac{\pi}{2}}\\[0.3cm]
                &= \frac{32}{5} ((1-\frac{2}{3}+\frac{1}{5})-(-1+\frac{2}{3}-\frac{1}{5}))
                = \frac{32}{5} \cdot 2(1-\frac{2}{3}+\frac{1}{5})\\[0.3cm]
                &= \frac{32}{5} \cdot \frac{8}{15} \cdot 2 = \frac{512}{75}
            \end{align*}
            \begin{center}
                \includegraphics[width=6cm]{img/ej5.png}
            \end{center}
	    }

        % Ejercicio 6
        \item {
            Calcular la masa del sólido que se encuentra fuera de la esfera
            $x^2+y^2+^z2=1$ y dentro de la esfera $x^2+y^2+z^2=2$ suponiendo
            que la densidad en un punto $P$ es directamente proporcional al
            cuadrado de la distancia de $P$ al centro de la esfera.

            \color{azul}
            Sabemos de la tarea anterior, que la masa de un sólido se calcula
            obteniendo la integral triple de la densidad del mismo. Sabemos que
            el centro de la esfera está en el origen, así tenemos que la
            densidad es la distancia $d(0,P) = \sqrt{x^2+y^2+z^2}$ por una
            constante $k$, ya que es directamente proporcional.
            
            De ahí, para obtener la masa tendríamos que obtener: 
            $\iiint_{W}k(x^2+y^2+z^2)dW$
            
            Transformamos a coordenadas esféricas, el jacobiano es
            $r^2\sen\theta$ y tenemos que $r^2 = x^2+y^2+z^2$
            Estamos considernado el solido entre dos esferas completas, así que
            podemos expresar la integral de la siguiente forma:		
				\[
                \int_{0}^{\sqrt{2}}{
                    \int_{0}^{2\pi}{
                        \int_{0}^{\pi}{
                           (kr^4\sen\gamma)
                        \,d\theta}
                    \,d\gamma}
                \,dr}
            \]
            
            Ya que las funciones son independientes unas de otras podemos
            resolver la integral como el producto de tres integrales
            independientes por una constante $k$:
			 \[
                (k)
                \left(\int_{0}^{\sqrt{2}}r^4 dr\right)
                \left(\int_{0}^{2\pi} d\theta\right)
                \left(\int_{0}^{\pi}\sen{\gamma} d\gamma\right)
                =
                (k)
                \left(\frac{r^5}{5}\Big|_0^{\sqrt{2}}\right)
                \left(\theta\Big|_0^{2\pi}\right)
                \left(-\cos{\gamma}\Big|_0^{\pi}\right)
                =
                (k)\left(\frac{2^{\frac{5}{2}} - 1}{5}\right)
                \left(2\pi\right)(2)
                =
               (4\pi k)\left(\frac{2^{\frac{5}{2}} - 1}{5}\right)
            \]           
           
           Por lo tanto, la masa del sólido es
           $\displaystyle (4\pi k)\left(\frac{2^{\frac{5}{2}} - 1}{5}\right)$
        }

        % Ejercicio 7
        \item {
            Determinar los números reales $\lambda$ para los que
            \[
                \iint_D {\frac{dA}{\left(x^2+y^2\right)^\lambda}}
            \]
            es convergente, con $D$ el disco unitario con centro en el origen.

            \color{azul}
            % Respuesta
            \begin{center}
                \includegraphics[width=4cm]{img/ej7.png}
            \end{center}
            La función $\displaystyle f(x)=\frac{1}{(x^2+y^2)^\lambda}$ no está
            definida cuando $x^2+y^2=0$, así que definimos una nueva región de
            integración $D_\varepsilon = \left\{(x,y)\in\mathbb{R}^2 |
            \varepsilon\leq x^2+y^2\leq 1\right\}$ y hacemos $\varepsilon
            \to 0$
            \[
                \iint_D {\frac{dA}{\left(x^2+y^2\right)^\lambda}}
                =\lim_{\varepsilon\to 0}{
                    \iint_{D_\varepsilon}{
                        \frac{dA}{\left(x^2+y^2\right)^\lambda}
                    }
                }
            \]
            para resolver la integral, conviene hacer una transformación a
            coordenadas polares
            \[
                \iint_D {\frac{dA}{\left(x^2+y^2\right)^\lambda}}
                =\int_{0}^{2\pi}{
                    \int_{0}^{1}{
                        \frac{r}{r^{2\lambda}}
                    \,dr}
                \,d\theta}
                =\lim_{\varepsilon\to 0}{
                    \int_{0}^{2\pi}{
                        \int_{\varepsilon}^{1}{
                            r^{1-2\lambda}
                        \,dr}
                    \,d\theta}
                }
            \]
            resolviendo la integral iterada (considerando $\lambda \ne 1$)
            \begin{align*}
                \lim_{\varepsilon\to 0}{
                    \int_{0}^{2\pi}{
                        \int_{\varepsilon}^{1}{
                            r^{1-2\lambda}
                        \,dr}
                    \,d\theta}
                }
                &=\lim_{\varepsilon\to 0}{
                    \int_{0}^{2\pi}{
                        \left[
                            \frac{r^{2-2\lambda}}
                                 {2-2\lambda}
                        \right]_{\varepsilon}^{1}
                    \,d\theta}
                }\\[.2cm]
                &=\lim_{\varepsilon\to 0}{
                    \int_{0}^{2\pi}{
                        \frac{1-\varepsilon^{2(1-\lambda)}}
                             {2(1-\lambda)}
                    \,d\theta}
                }\\[.2cm]
                &=\lim_{\varepsilon\to 0}{
                    \frac{2\pi[1-\varepsilon^{2(1-\lambda)}]}
                         {2(1-\lambda)}
                }\\[.2cm]
                &=\lim_{\varepsilon\to 0}{
                    \frac{\pi[1-\varepsilon^{2(1-\lambda)}]}
                         {1-\lambda}
                }
            \end{align*}
            Haciendo un análisis de ésta última expresión, concluimos que:
            \begin{itemize}
                \item si $\lambda>1$, entonces $1-\lambda>0$ y
                    $\varepsilon^{2(1-\lambda)}\to\infty$ (diverge)
                \item es cuando $\lambda<1$, entonces la integral converge.
            \end{itemize}

        }

        % Ejercicio 8
        \item {
            Calcular
            \[
                \iint_D {xye^{-\left(x^2+y^2\right)}\,dA}
            \]
            con $x\geq 0$ y $0\leq y\leq 1$.

            \color{azul}
            % Respuesta
            Notemos que la región de integración es infinita, como se ve en la
            siguiente figura
            \begin{center}
                \includegraphics[width=5cm]{img/ej8.png}
            \end{center}
            Así que la integral iterada a resolver es
            \[
                \int_{0}^{\infty}{
                    \int_{0}^{1}{
                        xye^{-(x^2+y^2)}
                    \,dy}
                \,dx}
            \]
            Sustituimos ese infinito con una nueva variable $t$ y hacemos que
            ésta tienda hacia infinito
            \begin{align*}
                \lim_{t\to\infty}{
                    \int_{0}^{t}{
                        \int_{0}^{1}{
                            xye^{-(x^2+y^2)}
                        \,dy}
                    \,dx}
                }
                &=\lim_{t\to\infty}{
                    \int_{0}^{t}{
                        x\left[-\frac{1}{2}e^{-(x^2-y^2)}\right]_0^1
                    \,dx}
                }\\[.3cm]
                &=-\frac{1}{2}\lim_{t\to\infty}{
                    \int_{0}^{t}{
                        x\left[e^{-(x^2+1)}-e^{-x^2}\right]
                    \,dx}
                }\\[.3cm]
                &=-\frac{1}{2}\lim_{t\to\infty}{
                    \int_{0}^{t}{
                        xe^{-x^2}\left[e^{-1}-1\right]
                    \,dx}
                }\\[.3cm]
                &=-\frac{e^{-1}-1}{2}\lim_{t\to\infty}{
                    -\frac{1}{2}\left[-\frac{1}{2}e^{-x^2}\right]_{0}^{t}
                }\\[.3cm]
                &=\frac{e^{-1}-1}{4}\lim_{t\to\infty}{
                    \left(e^{-t^2}-1\right)
                }
            \end{align*}
            Simplificamos la expresión $\frac{e^{-1}-1}{4}=\frac{1-e}{4e}$ y al
            calcular el límite $e^{-t^2}\to 0$ por lo que el límite es
            $-1$ y el resultado de la integral es $\displaystyle\frac{e-1}{4e}$
        }
    \end{enumerate}
\end{document}
