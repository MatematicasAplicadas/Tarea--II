% ====== TAREA 2 MATEMATICAS APLICADAS ======
\documentclass{article}
\usepackage[utf8]{inputenc}
\usepackage[spanish]{babel}
\usepackage{amsmath, amsfonts, amssymb}
\usepackage{graphics}
\usepackage[usenames]{color}
\usepackage[text={20cm,25cm},centering,top=1.5cm,bottom=1.5cm,letterpaper,showframe=false]{geometry}
\renewcommand{\baselinestretch}{1.5}
\parindent  = 0mm
\parskip    = 4mm
\definecolor{azul}{RGB}{10,80,190}
\definecolor{negro}{RGB}{0,0,0}
\definecolor{rojo}{RGB}{190,80,10}
\definecolor{verde}{RGB}{0,120,50}

\begin{document}
    \title{Tarea 2}
    \author{Careaga Carrillo Juan Manuel\\
            Quiróz Castañeda Edgar\\
            Soto Corderi Sandra del Mar}
    \date{Miércoles 10 de octubre de 2018}
    \maketitle
    \begin{enumerate}
    
        % Ejercicio 1
        \item {
            Encontrar la imagen de un triángulo con vértices $(0,0)$, $(1,1)$
            y $(0,1)$ bajo la transformación $x=u^2$ y $y=v$.

            \color{azul}
            Despejando $u$ de $x=u^2$, tenemos que $u=\sqrt{x}$ y que $v = y$\\
            De ahí, tendríamos que la transformación es:\\
            $T(x,y) = (\sqrt{x} , y)$\\
            
           	Aplicamos la transformación a sus vértices\\
            $T(0,0) = (\sqrt{0}, 0) = (0,0)$\\
            $T(1,1) = (\sqrt{1}, 1) = (1,1)$\\
            $T(0,1) = (\sqrt{0}, 1) = (0,1)$\\
            
            Por lo tanto, la imagen del triángulo es el mismo triángulo antes de ser                           transformado\\
          
            %%\begin{center}
            %% \includegraphics[width=8cm]{ejercicio1.png}
             %%\includegraphics[width=8cm]{ejercicio1-1.png}
        	%%\end{center}  
                	
        	En la imagen izquierda vemos el triángulo original, y en la derecha vemos 			la imagen del triángulo bajo la transformación.\\
	}        
   


        % Ejercicio 2
        \item {
            Calcular
            \[
                \iint_{D}{e^{\frac{x-2y}{3x-y}}\,dA}
            \]
            con $D$ el paralelogramo acotado por las rectas $x-2y=0$, $x-2y=$,
            $3x-y=1$ y $3x-y=8$-

            \color{azul}
            El paralelogramo $D$ lo podemos ver como la siguiente imagen:\\
            %%\begin{center}
            %% \includegraphics[width=8cm]{ejercicio2.png}
        	%%\end{center} 
        	
        	Para facilitar la integral, ralizaremos un cambio de variable y una transformación lineal con D.\\
        	Obtenemos el Jacobiano de la siguiente forma:\\
	$\begin{bmatrix}
    	\frac{\partial x}{\partial u}  & \frac{\partial x}{\partial v}\\
    	\frac{\partial y}{\partial u} & \frac{\partial y}{\partial v}\\
	\end{bmatrix}
	=
	\begin{bmatrix}
    	-\frac{1}{5}  & \frac{2}{5}\\
    	-\frac{3}{5} & \frac{1}{5}\\
	\end{bmatrix}
	=
	-\frac{7}{25}
	$      	
	
        	Para la transformación, tomamos $u = x-2y$ y $v= 3x-y$.\\ 
        	Despejando con un sistema de ecuaciones de u y v, tenemos que $x = -(\frac{u-2v}{5})$ y  $y = -(\frac{3u-v}{5})$ \\
        	
        	De ahí, si aplicamos la transformación en los lados del paralelogramo tenemos:\\
        	$T(x-2y=0) \Rightarrow (\frac{-u+2v}{5}) + 2(\frac{3u-v}{5}) = \frac{-u+2v+6u-2v}{5} = \frac{5u}{5} \Rightarrow u = 0$\\
        	$T(x-2y=4) \Rightarrow (\frac{-u+2v}{5}) + 2(\frac{3u-v}{5}) = \frac{-u+2v+6u-2v}{5} = \frac{5u}{5} \Rightarrow u = 4$\\
        	$T(3x-y=1) \Rightarrow -3(\frac{u-2v}{5}) + (\frac{3u-v}{5}) = \frac{-3u+6v+3u-v}{5} = \frac{5v}{5} \Rightarrow v = 1$\\
    		$T(3x-y=8) \Rightarrow -3(\frac{u-2v}{5}) + (\frac{3u-v}{5}) = \frac{-3u+6v+3u-v}{5} = \frac{5v}{5} \Rightarrow v = 8$\\
        	
        	La imagen del paralelogramo bajo la transformación es como la siguiente imagen, vemos que es un cuadrado, así que los límites de u y v son sencillos.\\
            %\begin{center}
             %%\includegraphics[width=8cm]{ejercicio2-1.png}
        	%%\end{center} 

	De ahí, podemos resolver a nuestra integral doble como:\\
	
	$\int_{1}^{8}\int_{0}^{4}-\frac{7}{25}e^{\frac{u}{v}}dudv$\\
	
	(Usando integración por sustitución tomando $s =\frac{u}{v}$, resolvemos directamente du) Tenemos:\\
	$= -\frac{7}{25} \int_{1}^{8} v(e^{\frac{4}{v}} -1)dv$\\
	$\int_{1}^{8} v(e^{\frac{4}{v}} -1)dv = \int_{1}^{8} ve^{\frac{4}{v}}dv - \int_{1}^{8} vdv$\\
	
	$\int_{1}^{8} vdv = \frac{v^2}{2}\Big|_1^8 = \frac{63}{2}$\\
	
	(Usando integración por sustitución tomando $t =\frac{4}{v}$,$dt = -\frac{v^2}{4}dv$ y $v= \frac{4}{t}$) Tenemos:\\
	$\int_{1}^{8} ve^{\frac{4}{v}}dv = -16\int_{1}^{8} \frac{e^{t}}{t^3}dt$\\
	(Usando integración por partes donde $u = e^t$ y $dv = \frac{1}{t^3})$)\\
	$\int_{1}^{8} \frac{e^{t}}{t^3}dt = -\frac{e^t}{2t^2} + \int-\frac{e^t}{2t^2}dt$\\
	(Usando integración por partes donde $u = e^t$ y $dv = \frac{1}{2t^2})$)\\
	$\int-\frac{e^t}{2t^2} = -\frac{1}{2} (-\frac{e^t}{t} + \int\frac{e^t}{t}dt)$\\
	(Usando la definición $Ei(t) = \int\frac{e^t}{t}dt$, resolvemos el resto de integral. También regresamos a los valores originales)\\
	$\int_{1}^{8} ve^{\frac{4}{v}}dv = [-16(-\frac{1}{32}e^{\frac{4}{v}}v^2 + \frac{1}{2}(-\frac{1}{4})e^{\frac{4}{v}}v + Ei(\frac{4}{v}))]\Big|_1^8$\\
	
	De todo lo anterior, el resultado sería:\\
	$\int_{1}^{8}\int_{0}^{4}-\frac{7}{25}e^{\frac{u}{v}}dudv = (-\frac{7}{25})(48\sqrt{e} - 8Ei(\frac{1}{2}) - 32 - \frac{5e^4 - 16Ei(4)}{2})$\\
	       
        }

        % Ejercicio 3
        \item {
            Hallar el volumen del elipsoide
            \[
                \frac{x^2}{a^2}+\frac{y^2}{b^2}+\frac{z^2}{c^2}\leq 1
            \]

            \color{azul}
            Hallar el volumen de forma típica nos daría una integral triple bastante compleja, así que para auxiliarnos, transformamos la elipsoide a coordenas esféricas y realizamos un cambio de variables.\\
           %\begin{center}
             %%\includegraphics[width=8cm]{elipsoide.png}
        	%%\end{center} 
        	
        	Tomamos las coordenadas esféricas:\\
        	$x = ar\sin\gamma\cos\theta$\\
        	$y = br\sin\gamma\sin\theta$\\
        	$x = cr\cos\gamma$\\
        	
        	El jacobiano sería la multiplicación de $abc$ con el jacobiano de coordenadas esféricas que vimos en clase es $r^2\sin\gamma$. De ahí, el jacobiano es $abcr^2\sin\gamma$\\
        	
        	El elipsoide es simétrico respecto al origen, los ejes y los planos de coordenadas. Además las secciones con planos paralelos a los coordenados son elipses (caso particular: circunferencia). De esta forma si vemos al elipsoide en coordenas esféricas podemos representar su volumen de esta manera:\\
        	$\int_{0}^{1}\int_{0}^{2\pi}\int_{0}^{\pi}abcr^2\sin\gamma d\gamma d\theta dr$\\
        	
        	Debido a que las variables son independientes una de otra podemos resolver la integral así:\\
        	$(abc)(\int_{0}^{1}r^2 dr)(\int_{0}^{2\pi} d\theta)(\int_{0}^{\pi}\sin\gamma d\gamma) = (abc)(\frac{r^3}{3}\Big|_0^1)(\theta\Big|_0^{2\pi})(-\cos\gamma\Big|_0^{\pi}) = (abc)(\frac{1}{3})(2\pi)(2) = \frac{4}{3}\pi abc$\\
        	
        	Por lo tanto, el volumen del elipsoide es $\frac{4}{3}\pi abc$.\\
        	

        }

        % Ejercicio 4
        \item {
            Hallar el área acotada por la lemniscata
            \(
                \left(x^2+y^2\right)^2=2a^2\left(x^2-y^2\right)
            \).

            \color{azul}
            % Respuesta
        }

        % Ejercicio 5
        \item {
            Evaluar la integral iterada
            \[
                \int_{0}^{2}{
                    \int_{-\sqrt{2x-x^2}}^{\sqrt{2x-x^2}}{
                        \int_{0}^{x^2+y^2}{
                            \sqrt{x^2+y^2}
                        \,dz}
                    \,dy}
                \,dx}
            \]
            y bosquejar la región de integración $W$.

            \color{azul}
            % Respuesta
        }

        % Ejercicio 6
        \item {
            Calcular la masa del sólido que se encuentra fuera de la esfera
            $x^2+y^2+^z2=1$ y dentro de la esfera $x^2+y^2+z^2=2$ suponiendo
            que la densidad en un punto $P$ es directamente proporcional al
            cuadrado de la distancia de $P$ al centro de la esfera.

            \color{azul}
            % Respuesta
        }

        % Ejercicio 7
        \item {
            Determinar los números reales $\lambda$ para los que
            \[
                \iint_D {\frac{dA}{\left(x^2+y^2\right)^\lambda}}
            \]
            es convergente, con $D$ el disco unitario con centro en el origen.

            \color{azul}
            % Respuesta
        }

        % Ejercicio 8
        \item {
            Calcular
            \[
                \iint_D {xye^{-\left(x^2+y^2\right)}\,dA}
            \]
            con $x\geq 0$ y $0\leq y\leq 1$.

            \color{azul}
            % Respuesta
        }
    \end{enumerate}
\end{document}
